\chapter{The informatics of biology}
Bioinformatics is the application of computational methods to biological problems,
either by analysing large amounts of data, or by replacing experiments with computer programs.

One of the main goals of protein bioinformatics is predicting the 3D structure of a protein given only its sequence.

\section{Predictors}

\section[Multiple Sequence Alignments]{Increasing statistics: Multiple Sequence Alignments}

\section{Contact prediction}

A Multiple Sequence Alignment (MSA) contains a collection of proteins evolutionary related to our query, 
\marginpar{Correlated mutations}

\subsection{Mutual Information}

\subsubsection{Phylogenetic bias: the APC correction}

\subsection{Direct Coupling Analysis}
Mutual Information has a problem with the transitivity property: \todo{}
In order to discern true from spurious correlations, we need to fit a statistical model to the whole data at once, in this way, we hope to recover the true (direct) relationships.
This can be accomplished with Direct Coupling Analysis (DCA).
Several variations exists, but they are all based on a Potts model of statistical mechanics.
This is a model where each position (in our case, residue), can take a number of discrete, well defined, spin states (amino acid types), and the model depends only on the intrinsic properties at each location, and pairwise interactions.

The energy of a sequence of amino acids $\vec{\sigma}$ takes the form:
\begin{equation*}
H(\vec \sigma) = \sum_{\substack{i,j=1\\i \neq j}}^N J_{i, j}(\sigma_i, \sigma_j) + \sum_{i=1}^N h_i(\sigma_i),
\end{equation*}
where $h_i(\sigma_i)$ is the chemical potential of having a given amino acid at position $i$, and $ J_{i, j}(\sigma_i, \sigma_j)$ is the pairwise interaction between residues $i$ and $j$ given their amino acid species.

We can assume the proteins in our MSA were generated by a similar model, so their frequency should follow the Boltzmann distribution:

\begin{equation*}
p(\vec{\sigma} |  h, J) = \frac{1}{Z} e^{-\beta H\left(\vec{\sigma}\right)}
\end{equation*}
$\beta$ is a scaling factor, inverse of the temperature, and $Z$ is the partition function, a normalisation term to ensure all the probabilities sum up to 1:

\begin{equation*}
Z = \sum_{\vec{\sigma}} e^{-\beta H\left(\vec{\sigma}\right)}
\end{equation*}

Fitting a Potts model means estimate the values of $J$ and $h$ that best explain the observed distribution of sequences in the MSA.
The problem as such is intractable because computing $Z$ implies a sum over all the possible sequences of length $N$, $21^N$ (the 20 natural amino acids plus the gap state).

There are several strategies that can approximate the partition function, such as plmDCA \citep{plmDCA}, or side-step it all together, like GaussDCA \citep{GaussDCA}.

Once the values of $J$ are obtained, the scores of the contacts can be estimated by taking the Frobenius norm of each of the J matrices.
That is, the square root of the sum of the squares of the couplings between each pair of amino acids:

\begin{equation*}
C(i, j) = \sqrt{\sum_{\sigma_i, \sigma_j=1}^{21} J_{i, j}(\sigma_i, \sigma_j)^2},
\end{equation*}
where $C(i, j)$ is the contact score between residues $i$ and $j$.
This number cannot be readily interpreted as a probability.

This model approximations ignore the possibility of multiple rotamers for a given amino acid, \marginpar{The sphericity of the cow}
and consider that any interaction between three or more residues can be decomposed to the sum of each pair.
Finally, DCA attempts to reconstruct the evolutionary couplings, not necessarily the contacts.
\citet{contact_errors} showed that most of the top-scoring pairs indicated by DCA are true contacts, others correspond to contacts between different subunits in a homodimer or pairs of separated residues involved in the function.
The coupling is real, but its nature is different from the model explained at the beginning of this section.

\subsection{Pattern recognition}

\section{Protein folding}


\subsection{Homology modelling}
\subsection{\emph{Ab-initio} folding}

\section{Other predictors}

