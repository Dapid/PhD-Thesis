\chapter*{Sammanfattning}
\lettrine[lines=3, lhang=0.25, nindent=0em, findent=2pt]{\color{Maroon}P}{roteiner är\ } grundläggande molekulära maskiner i cellen. Experimentella studier är oftast arbetskraftsintensiva och dyra; och många av dem misslyckas utan att ge resultat. Ett av målen av bioinformatik är att, om möjligt, ersätta dessa experiment med beräkningsmetoder.

Den mänskliga impulsen är att kategorisera och modellera allt baserat på hårda och enkla regler.
Men naturen är rörig och motstår att följa dessa regler.
En av lösningarna som lades fram var användningen av maskininlärning: mjuka och komplicerade regler, som datorn når efter att ha tittat på data.
Reglerna är `` mjuka '' eftersom de är probabilistiska och 'komplicerade' eftersom de är resultatet av att analysera stora datasätt.

Under det senaste decenniet har djupinlärning revolutionerat området maskininlärning, särskilt inom datorsyn och taligenkänning.Det främsta skälet till dess framgång är förmågan att träna mycket flexibla modeller, som kan fånga den verkliga världen, på stora mängder data. Men nyckeln som gör det möjligt är förmågan att utnyttja strukturen i data.

Detta arbete presenterar tillämpningen av maskininlärning i allmänhet och djupinlärning i synnerhet på flera uppgifter inom området för förutsägelse av proteinstrukturer: kontaktprognos, \ emph {ab-initio} modellering, modellkvalitetsbedömning.

Fokus för min forskning har varit att undersöka hur man kan representera proteindata på ett sätt som är lämpligt för djupinlärning. Mer konkret har jag arbetat med att utveckla metoder som är effektiva men ändå enkla att installera och använda.