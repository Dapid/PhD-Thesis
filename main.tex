%\documentclass[a5paper, 10pt, twoside, openany]{book}
\documentclass[fontsize=9pt,paper=a5,footinclude,headinclude, twoside, final]{scrbook} % KOMA-Script book
\usepackage[cmyk,hyperref,dvipsnames]{xcolor} %The two first options are for pdf-x, the last is for classicthesis

\usepackage[pdfpagelabels, unicode, pdfversion=1.5]{hyperref}
\usepackage[swedish,spanish,british]{babel}
\usepackage{url}
\usepackage{iftex}

\usepackage{adforn}

\usepackage[centertags]{amsmath}
%\usepackage{amsfonts}
\usepackage{amssymb}
\usepackage{amsthm}

\ifLuaTeX
	\usepackage{fontspec}
\else
	\usepackage[utf8]{inputenc}
	\usepackage[T1]{fontenc}
\fi

\usepackage{graphicx}
\usepackage{caption}
\usepackage{subcaption}
\usepackage{bibentry}

\usepackage{siunitx}
\usepackage{fancyvrb}
\usepackage{lettrine}

% Remove indentation in bulletpoints
\usepackage{enumitem}
\setlist{leftmargin=0mm}

\ifLuaTeX
	\usepackage[math-style=french]{unicode-math}
\fi
\usepackage{ebgaramond} % To get the swash

\usepackage[eulerchapternumbers=true, eulermath=true, parts=true, floatperchapter]{classicthesis}
\usepackage[paperwidth=165mm, paperheight=242mm, outer=40mm, inner=20mm, marginparwidth=25mm, top=20mm]{geometry}
%\usepackage[paperwidth=165mm, paperheight=242mm]{geometry}
%\usepackage{Baskervaldx}
%\usepackage[p,osf]{baskervillef}

%\ifLuaTeX
%	\setmathfont[Scale=MatchUppercase]{Asana Math}
%\fi

%\usepackage[backend=biber, style=mla, natbib=True]{biblatex}
%\addbibresource{references.bib}
\usepackage[authoryear, round]{natbib} % Allow to split citations across lines.
\bibliographystyle{abbrvnat}
\usepackage{bibentry}
\nobibliography*

\usepackage{enumitem}

\usepackage{xpatch} %Adjust the spacing in smallcaps
\xapptocmd{\scshape}{\spaceskip=2\fontdimen2\font plus 2\fontdimen3\font minus \fontdimen4\font	\xspaceskip=0\fontdimen7\font}{}{}

\usepackage{tikz}
\usetikzlibrary{calc,shapes,decorations}

\usepackage{microtype}
\SetProtrusion{encoding=*, family=*}{- = {, 1000}} % Hanging hyphens
\SetProtrusion{encoding=*, family=palatino-it}{- = {, 1000} } % Hanging hyphens

%\usepackage{placeins}
%\usepackage{todonotes}

\usepackage[hyphenation,lastparline,nosingleletter,homeoarchy]{impnattypo} %draft
\usepackage[defaultlines=2]{nowidow}
\setnowidow

\frenchspacing   % Do not include extra space after sentences.

\usepackage[x-4]{pdfx} % X is printing standard. 5g is the latest supported version.

\hyphenation{PSICOV, PconsC3, PconsFold, PconsC2, PconsFold2}
\DeclareMathOperator{\arctantwo}{arctan2}

%\setcounter{tocdepth}{1} % Show sections
\setcounter{tocdepth}{2} % + subsections

%Set itemize to raggedright
\setlist[itemize]{before=\csname par\endcsname\raggedright,	partopsep=0pt}

\newcommand{\MSA}{\textsc{msa}}
\newcommand{\HMM}{\textsc{hmm}}
\newcommand{\HMMs}{\textsc{hmm}s}
\newcommand{\PSSM}{\textsc{pssm}}
\newcommand{\PSSMs}{\textsc{pssm}s}
\newcommand{\JackHMMER}{Jack\textsc{hmmer}}
\newcommand{\HHBlits}{\textsc{hhb}lits}
\newcommand{\PSIBLAST}{\textsc{psi-blast}}
\newcommand{\BLAST}{\textsc{blast}}
\newcommand{\DCA}{\textsc{dca}}
\newcommand{\GaussDCA}{Gauss\DCA}
\newcommand{\plmDCA}{plm\DCA}
\newcommand{\APC}{\textsc{apc}}
\newcommand{\MODELLER}{\textsc{modeller}}
\newcommand{\CONFOLD}{\textsc{confold}}
\newcommand{\CNS}{\textsc{cns}}
\newcommand{\RMSD}{\textsc{rmsd}}
\newcommand{\LDDT}{\textsc{lddt}}
\newcommand{\CAD}{\textsc{cad}}
\newcommand{\MQA}{\textsc{mqa}}
\newcommand{\EMA}{\textsc{ema}}
\newcommand{\PDB}{\textsc{pdb}}
\newcommand{\ADADELTA}{\textsc{adadelta}}
\newcommand{\CASP}{\textsc{casp}}
\newcommand{\DNA}{\textsc{dna}}
\newcommand{\RNA}{\textsc{rna}}
\newcommand{\NMR}{\textsc{nmr}}
\newcommand{\EM}{\textsc{em}}
\newcommand{\CCD}{\textsc{ccd}}
\newcommand{\SAXS}{\textsc{saxs}}
\newcommand{\SANS}{\textsc{sans}}
\newcommand{\MSE}{\textsc{mse}}
\newcommand{\LASSO}{\textsc{lasso}}
\newcommand{\OLS}{\textsc{ols}}
\newcommand{\SVM}{\textsc{svm}}
\newcommand{\RBF}{\textsc{rbf}}
\newcommand{\GP}{\textsc{gp}}
\newcommand{\RNN}{\textsc{rnn}}
\newcommand{\MLP}{\textsc{mlp}}
\newcommand{\LSTM}{\textsc{lstm}}
\newcommand{\ELU}{\textsc{elu}}
\newcommand{\TM}{\textsc{tm}}
\newcommand{\Ss}{\textsc{s}}
\newcommand{\RSA}{\textsc{rsa}}
\newcommand{\SGD}{\textsc{sgd}}
\newcommand{\CNN}{\textsc{cnn}}


\setsansfont[Ligatures=TeX]{Cabin}
\newcommand{\sidenote}[1]{\marginpar{\normalsize #1}}
\newcommand{\longsidenote}[1]{\marginpar{\footnotesize\upshape\sffamily #1}}


\begin{document}

\author{David Menéndez Hurtado}
\title{\textsc{Structured learning for structural bioinformatics}}
\date{2019}

\frontmatter
\begin{titlepage}
	\begin{addmargin}[-1cm]{-3cm}
		\begin{center}
			\large
			
			\hfill
			
			\vfill
			
			\begingroup
			\Huge
			\color{Maroon}\textsc{ Structured Learning \\ for
				\\ Structural Bioinformatics} \\ \bigskip
			\endgroup
			
			\vfill
			 
			{\huge David Menéndez Hurtado}
			
			\vfill
			
			
			
			%\mySubtitle \\ \medskip
			PhD thesis \\ \medskip
			Department of Biochemistry and Biophysics\\ \bigskip
			Stockholm University \\ \bigskip
			%\myUni \\ \bigskip
			
		
			
			\vfill
			
		\end{center}
	\end{addmargin}
\end{titlepage}

{\large
\KOMAoptions{open=left}
\chapter*{Abstract}
\lettrine[lines=3, lhang=0.25, nindent=0em, findent=2pt]{\color{Maroon}P}{roteins are\ }
the basic molecular machines of the cell.
Experimental studies are usually labour-intensive and expensive; and in no few cases, fail without giving a result.
One of the aims of the field of Bioinformatics is to, when possible, replace these experiments with computational models.

The human impulse is to categorise and model everything based on hard and simple rules.
But Nature is messy and resists to adhere to these rules.
One of the solutions put forward was the use of machine learning: soft and complicated rules, reached by the computer after looking at the data.
The rules are ``soft" because they are probabilistic in nature, and ``complicated" because they are the result of analysing large datasets.

In the last decade, deep learning has revolutionised the field of machine learning, in particular in computer vision and speech recognition.
The primary reason for their success is the ability to train very flexible models, capable of capturing the complexity of the real world, on large amounts of data.
But the key that makes it feasible is the ability to leverage the structure in the data. 

This work presents the application of machine learning in general, and deep learning in particular, to several tasks in the field of protein structure prediction: contact prediction, \emph{ab-initio} modelling, model quality assessment.

The focus of my research has been exploring how to represent the protein data in a way that is suitable for deep learning.
On more practical matters, I have worked on the development of methods that are effective yet easy to install and use.

\chapter{Sammanfattning}
Djup lärning
}

{\raggedright
\chapter*{Papers included in this thesis}

\section*{Paper \textcolor[cmyk]{0, 0.87, 0.68, 0.32}{I}}
Karolis Uziela, Nanjiang Shu, Björn Wallner, and Arne Elofsson. ProQ3: Improved model quality
assessments using rosetta energy terms. \textit{Scientific Reports}, 6:33509, Oct 2016. doi:10.1038/srep33509.
URL \url{https://dx.doi.org/10.1038/srep33509}.

\section*{Paper  \textcolor[cmyk]{0, 0.87, 0.68, 0.32}{II}}
Mirco Michel, David Menéndez Hurtado, Karolis Uziela, and Arne Elofsson. Large-scale structure
prediction by improved contact predictions and model quality assessment. \textit{Bioinformatics}, 33(14):
i23–i29, 07 2017a. ISSN 1367-4803. doi:10.1093/bioinformatics/btx239. URL
\url{https://doi.org/10.1093/bioinformatics/btx239}.


\section*{Paper \textcolor[cmyk]{0, 0.87, 0.68, 0.32}{III}}
Mirco Michel, David Menéndez Hurtado, and Arne Elofsson. PconsC4: fast, accurate and hassle-free
contact predictions. \textit{Bioinformatics}, 35(15):2677–2679, 12 2018. ISSN 1367-4803.
doi:10.1093/bioinformatics/bty1036. URL \url{https://doi.org/10.1093/bioinformatics/bty1036}.


\section*{Paper \textcolor[cmyk]{0, 0.87, 0.68, 0.32}{IV}}
David Menéndez Hurtado, Karolis Uziela, and Arne Elofsson, A novel training procedure to train deep networks in the assessment of the quality of protein models. \emph{Manuscript}


\section*{Paper \textcolor[cmyk]{0, 0.87, 0.68, 0.32}{V}}

Jianlin Cheng, Myong-Ho Choe, Arne Elofsson, Kun-Sop Han, Jie Hou, Ali H. A. Maghrabi, Liam J.
McGuffin, David Menéndez-Hurtado, Kliment Olechnovič, Torsten Schwede, Gabriel Studer, Karolis
Uziela, Česlovas Venclovas, and Björn Wallner. Estimation of model accuracy in CASP13. \emph{Proteins:
Structure, Function, and Bioinformatics}, in press, 2019. doi:10.1002/prot.25767.


}

\KOMAoptions{open=left}
{ %\small
	\large
	 \tableofcontents}

\KOMAoptions{open=left}
\setlist{leftmargin=15mm}
\chapter{Introduction}
\todo[inline]{Motivations of the thesis}

This thesis is divided into three parts:

\begin{itemize}
	\item[Part \ref{part:info}] introduces machine learning, the workhorse of my work.
	\item[Part \ref{part:bio}] explains the biological and biochemical underpinnings.
	\item[Part \ref{part:work}] summarizes the content of each paper in this thesis.
\end{itemize}

Following are the papers as published in the journals.

\setlist{leftmargin=0mm}

\KOMAoptions{open=right}
\mainmatter 

\part{The informatics} \label{part:info}
\chapter{Machine learning}
%\section{What is machine learning?}

The Scientific Revolution during the Renaissance \marginpar{A historical note}
was fuelled by the realisation that Nature can be parametrised using only a handful of equations.
With these tools, and sufficient measurements, natural philosophers could use the universal equations to fully model the mechanical universe.
As mathematical tools improved, increasingly more complex systems could be analysed, such as orbits of astrophysical objects or complex optical designs.
Furthermore, the development of computers allowed for an explosion in the size and breadth of tractable problems, such as dynamics of atomic nuclei, or weather forecast.

These applications were either developed from first principles, or as effective theories; but contained at their core, relatively simple mathematical models.
For example, modelling the dynamics of $n$-bodies gravitationally interacting are very rich and complicated to solve, but they all spawn from two simple equations: $\vec{F} = m \cdot \vec{a}$  and $F_g=G\frac{m_1  m_2}{r^2}$.

But the advent of computers not only gave us mathematical muscle, it provided us with large amounts of data on the real world.
Data that can be used to fit statistical models, even in the absence of underlying mathematical theories, such as recognising objects in images, or translating natural language.
The Scientific Revolution of the 16th century brought the concept \emph{``if it can be measured, it can be modelled"}, but the Data Revolution of the 20th century expanded it \emph{``if it can be \emph{represented}, it can be modelled"}.

Machine learning \marginpar{What is machine learning?}
is the study of statistical models that can create inferences from collections of examples. 
The models can be as concrete as relating the voltage and measured current intensity in a circuit, as complex as relating the sensory input of a rocket with its control, or as abstract as mapping natural images to the text describing its contents.



In general, designing a computer program fully capable of solving these problems can be time-consuming, and with a high degree of complexity.
Machine learning solves this limitation with data: instead of a programmer deciding every step, the algorithm depends on a series of free parameters that are deduced from the data.

\section{Classification and typology}
Machine learning tasks can be classified according to several criteria.
Here are, in broad strokes, some of the main types that cover the majority of the machine learning problems according to different criteria.

\subsection{Do we have labels?}
\begin{itemize}
\item \emph{Supervised:} our training data has assigned labels, and we want to predict them to new data. \emph{Ex: image recognition, linear regression.}
\item \emph{Unsupervised:} we do not have data with annotated target values. \emph{Ex: clustering, dimensionality reduction.}
\end{itemize}

The focus of this thesis will be on supervised tasks.

\subsection{Are labels categorical or continuous?}
The supervised tasks can be again divided depending on the nature of the labels:
\begin{itemize}
\item \emph{Classification:} our labels are categorical variables. \emph{Ex: image recognition, automated transcription of speech, presence or absence of tumours}
\item \emph{Regression:} we are interested in the value of continuous variables. \emph{Ex: curve fitting, counting.}
\end{itemize}

\section{The machine learning spectrum}
We can design machine learning models with different degrees of restrictions, or parametric assumptions.
A more restricted model needs less data to converge, and its performance will not be hindered if the underlying assumptions are correct.
On the other hand, if these restrictions are not accurate, the model will be biased and its performance, limited.

If we instead relax the parametric assumptions we obtain a more flexible model, capable of tackling more complex problems.
But this versatility comes with a cost: they require more data to train.

\begin{center}
	\missingfigure[figcolor=white]{Spectrum diagram}
\end{center}


Can we take it to the extreme?
\marginpar{A theoretical result} 
Can we train a model completely free  of assumptions in the case of infinite data? The No Free Lunch Theorem \citep{no_free_lunch} says, averaging over all problems, all algorithms are equally good.
In other words, without inputting domain knowledge, we cannot do better than random.

\section{A point of comparison: traditional machine learning}
In this section, I will give an overview to some of the most popular supervised machine learning algorithms to illustrate how they work, and the kind of underlying assumptions they operate under.

\subsection{Ordinary Least Squares, Ridge, and LASSO}
\subsection{Suport Vector Machines}
\subsection{Decision tree}
\subsection{Random Forest}
\subsection{Gaussian Processes}
\subsection{Monotonic regression}

\section{On the wrongness of machine learning}


\chapter{Deep learning}
\section{Success stories}
We start this section introducing the two fields that have been spearheading the Deep Learning revolution: computer vision and speech recognition.
The first is 
\todo{perceptual learning}

\section{The basic blocks}
\section{Taming the complexity: regularisation}
Neural networks can have millions of parameters, so they are susceptible to over-fitting.
In order to converge to generalisable models, we can apply a variety of regularisation techniques.

In general, they are a barrier that hinders the training, that will only be overcame if enough data supports it.
Here are some:

\marginpar{Weight decay}
To prevent any single activation from

$L^2$ regularisation can be interpreted as a Gaussian prior over the weights centred around $0$.

The most popular technique \marginpar{Dropout} 
specifically developed for deep learning is Dropout, \citep{dropout}. 
During training, a random fraction $0 < \rho < 1$ of intermediate inputs is set to $0$, while the rest of values are scaled by a factor of $\frac{1}{1-\rho}$ to compensate.

Since 

\marginpar{Batch Normalisation}

\marginpar{Differential privacy}
Special mention

\marginpar{Architectural}

\section{The quest for depth}
b
\section{Deep transfer learning}
a




%Practical advice for DL practice

\part{The biology} \label{part:bio}
\chapter{Proteins}
\section{What are proteins and why do we care?}
Proteins are the fundamental machines of biology, performing tasks such as catalysis, transporting molecules, or providing the structural backbone of the cell.
They play fundamental roles in the biochemical pathways, so understanding them can lead us to create new and refined drugs,
or bio-engineered bacteria.

%In this chapter I will present the basic vocabulary of protein structures, and give an overview of their biochemistry.

But, how are they created?
The information \marginpar{Protein biosynthesis}
used by the cell to build proteins is encoded in the DNA.
When the biosynthesis begins, the double helix unfolds, and the genetic contact is translated into RNA.
This new molecule is then transported to the ribosomes, the organelles that translate the RNA into functional proteins.


\section{The many levels of protein structures}
Proteins are structured at several levels, depending at the scale we look at them.
In this section, I will present the main descriptions of proteins at different scopes.


\subsection{The amino acids}
Amino acids are the building blocks of proteins.
They are composed of a carboxyl (-$COOH$) and an amine (-$NH_2$) groups, forming the backbone, and a side-chain.
Two examples are illustrated in Figure~\ref{fig:amino_acids}.
\marginpar{Only a few of all amino acids appear in the DNA.}
There are at least 500 known naturally occurring amino acids \citep{500_amino_acids}, of which only twenty different species are usually codified in the DNA.
The differences between most of them are only on the side chain\footnote{Proline's sidechain has a ring connecting with the amine in the backbone.}.

The side chains are the group responsible for the specific physico-chemical properties of the compound, such as hydrophobicity, pH, or electrostatic charge.

The backbone can polymerise, bonding with other amino acids and forming a long chain.
Having a common backbone means that in principle, any amino acid can be connected to any other, which gives proteins a very rich and flexible biochemistry:
for a protein of length $L$ there are $20^L$ possible combinations. %$20/19 (20^L - 1)$

\begin{figure}
\centering
\hfil %
\subcaptionbox{Glutamine (Q) \label{subfig:aminoQ}}{\input{bioinfo/figures/aminoQ.tikz}} %
\hfil %
\subcaptionbox{Tyrosine (Y) \label{subfig:aminoY}}{\begin{tikzpicture}
    \node[anchor=south west,inner sep=0] (image) at (0,0) {\includegraphics[width=0.45\columnwidth]{bioinfo/figures/amino_acid_Y}};
    \begin{scope}[x={(image.south east)},y={(image.north west)}]
	    \draw[Maroon, thick] (-0.05, -0.05) rectangle (0.6,0.3);
 	    \draw[RoyalBlue, thick, dashed] (0.2, 0.35) rectangle (1.05, 1.05);
        %\draw[help lines,xstep=.1,ystep=.1] (0,0) grid (1,1);
        %\foreach \x in {0,1,...,9} { \node [anchor=north] at (\x/10,0) {0.\x}; }
        %\foreach \y in {0,1,...,9} { \node [anchor=east] at (0,\y/10) {0.\y}; }
    \end{scope}
\end{tikzpicture}
} %
\hfil %
\caption{Two amino acids as appear in proteins.
The \textcolor{Maroon}{maroon}, solid rectangle indicates the backbone, common to all amino acids; and the \textcolor{RoyalBlue}{blue} dashed the side chain, that determines the specific chemical properties.}\label{fig:amino_acids}
\end{figure}

\subsection{Primary structure}
The amino acids can connect to each other through the peptide bond forming a chain.
The \emph{primary structure} is the sequence of amino acids as encoded in the DNA:
\begin{center}
\texttt{ALA ARG ILE ASN GLY ARG GLU ILE ASN VAL THR LYS LYS}
\end{center}

This is the easiest to obtain experimentally, since the advent of next generation sequencing techniques.
The collection of all protein sequences of an organism is the \emph{proteome}.


\subsection{Secondary structure}
The polypeptide chains are locally organised in motifs stabilised by hydrogen bonds.
The most common is the $\alpha$-helix, \marginpar{$\alpha$} shown in Figure~\ref{subfig:alpha}, where the hydrogen bonds are formed between the backbone oxygen of one residue, and the hydrogen of the amine group, four residues beyond, and continued in a regular pattern, forcing the backbone to twist into a helix.
The same bond is possible with residues that are closer -- two or three residues -- or further  -- five residues, the $\pi$ helix -- but are less common.

The second most common arrangement is the $\beta$-sheet  \marginpar{$\beta$} depicted in Figure~\ref{subfig:beta}, where approximately extended chains are placed next to each other and bonds are formed between juxtaposed residues.

\begin{figure}[hb]
	\centering
	\subcaptionbox{$\alpha$-helix\label{subfig:alpha}}{\includegraphics[width=0.9\textwidth]{bioinfo/figures/helix}}\\
	\subcaptionbox{$\beta$-sheet\label{subfig:beta}}{\includegraphics[width=0.9\textwidth]{bioinfo/figures/beta}}
	\caption{The two most frequent secondary structure elements.}\label{fig:alpha_beta}
\end{figure}


\subsection{Tertiary structure}
Once the chain is locally stabilised by the hydrogen bonds it needs to fold into a compact structure.
The spatial arrangement of the secondary structure elements is the \emph{tertiary structure.}
One example is the chain A of the protein \texttt{4V0B}, in Figure~\ref{fig:tertiary}.

\begin{figure}[htb]
\centering
\includegraphics[width=0.8\textwidth]{bioinfo/figures/tertiary}
\caption{Tertiary structure of the chain B of the protein \texttt{4V0B}.}\label{fig:tertiary}
\end{figure}

\subsection{Quaternary structure}
Proteins do not always work alone, but they form complexes composed of several chains.
The relative arrangement of each chain is the quaternary structure.

\begin{figure}[htb]
\centering
\includegraphics[width=0.8\textwidth]{bioinfo/figures/quaternary}
\caption{Quaternary structure of the chains A and B of the protein \texttt{4V0B}.}\label{fig:quaternary}
\end{figure}

%\subsection{Quintenary structure} %?
\section{Protein biochemistry}

Experiments show that proteins placed in the right medium of pH and temperature will unfold into a random coil, with no trace of tertiary nor secondary structure.
When restored to physiological conditions, they will refold into its natural shape, recovering its original biological and chemical properties.
As long as the protein is chemically unaltered, it will recover the tertiary structure independently of the folding machinery of the cell.
\marginpar{Dogmas that govern protein folding}
This was codified as Anfinsen's dogma: under physiological conditions, the primary structure of globular proteins determines the secondary and tertiary structure, and thus, its function, independently of the cell's machinery \citep{Anfinsen_dogma}.
Typically, the fully folded state is reached in the order of mili- to seconds.
The hypothesis has a few exceptions, namely large, fibrous proteins, prions -- proteins with alternative, stable conformations -- and aggregating proteins -- where the individual proteins bind to each other forming large and non-functional structures.
A reasonable hypothesis is that, a priori, the native conformation seems to be the state in the lowest energy, and the random coil is just rolling down the energy landscape.

\citet{fold_graciously} \marginpar{Levinthal's paradox} noted that, for every amino acid, a protein has two main degrees of freedom corresponding to the torsion of the backbone, plus one more for the rotation of the side chain.
This gives us a configuration space of the order of $10^{2L}$ for a protein of $L$ residues, but the kinematics suggest that it only has time to sample $10^8$ conformations, an exponentially tiny fraction for proteins of typical length.

Levinthal suggested two corrections:
\begin{itemize}
	\item The native conformation is not necessarily the one of minimum energy, but a local minimum with a deep enough well that it is stable.
	\item The native conformation must be kinematically accessible, possibly guided by local interactions that partially fold the protein.
\end{itemize}

These corrections are supported by experiments showing that a certain enzyme only folds at temperatures around 37 C, but it completely stable up to 90 C.
Furthermore, the same enzyme produced by an organism that lives at colder temperatures is shown to require lower temperatures for renaturation, but it is still stable up to 90 C.
The order of events greatly influences the kinetics of folding.

\FloatBarrier
%\subsection{The protein backbone}
\subsection{The hydrophobic effect}

\subsection{Energy terms}
A relatively generic form of the energy dictating the dynamics of a protein is:
\begin{align*}
H(\left\{\vec r_i\right\}_{i=0}^N) =& \underbrace{\sum_{bonds} k_b \left(d - d_0\right)^2}_\alpha + 
\underbrace{\sum_{angles} k_a \left(\theta - \theta_0\right)}_\beta +  \nonumber \\
&+ \underbrace{\sum_{torsions} f\left(\omega\right)}_\gamma +
\underbrace{\sum_{free\ pairs} k_{ij} \left[\left(\frac{r_{0ij}}{r_{ij}}\right)^{12} - 2 \left(\frac{r_{0ij}}{r_{ij}}\right)^{6} \right]}_\delta + \nonumber \\
&+ \underbrace{\sum_{i,j} \frac{q_i q_j}{4 \pi \epsilon r_{ij}}}_\epsilon
\end{align*}
The terms correspond to:

\begin{itemize}
\item[$\alpha$] Harmonic potential on the bond lengths $d$ for every pair of covalent bonded atoms, where $d_0$ is the ideal bond length and $k_b$ is the strength of the potential for the atom types.
\item[$\beta$] Harmonic potential on the angles between adjacent bonds $\theta$.
\item[$\gamma$] Arbitrary function $f$ on every torsion angle $\omega$.
\item[$\delta$] Lennard-Jones potential between all pairs of atoms not covalent bonded, where $r_{0ij}$ denotes the equilibrium distance and $k_{ij}$ the depth of the energy well.
\item[$\epsilon$] Electrostatic energy between charges, assuming they are spherically distributed.
\end{itemize}

Note that the collection $\vec r_i$ % $\left\{\vec r_i\right\}_{i=0}^N$
includes the water molecules and other atoms in the environment of the protein itself.

Of these terms, $\alpha$ and $\beta$ are strictly valid only near their respective minima, while $\delta$ is valid for distances around and beyond the minimum.
In practice, this is not a significant problem because the deviations from the ideal values are small. \marginpar{Dynamics}
From here we can see that most of the dynamics in proteins is given by the flexibility of the torsion angles.
Roughly speaking the torsions on the backbone determine the secondary structure, and the torsions of the side chains determine the packing.

\section{Membrane proteins}

\section{Experimental determination}

\chapter{The informatics of biology}
Bioinformatics is the application of computational methods to biological problems,
either by analysing large amounts of data, or by replacing experiments with computer programs.

One of the main goals of protein bioinformatics is predicting the 3D structure of a protein given only its sequence.

\section{Predictors}

\section[Multiple Sequence Alignments]{Increasing statistics: Multiple Sequence Alignments}

\section{Contact prediction}

\marginpar{Correlated mutations}

\subsection{Mutual Information}

\subsection{Direct Coupling Analysis}

\begin{equation*}
H(\vec \sigma) = \sum_{i,j=1, i!=j}^N J_{i, j}(\sigma_i, \sigma_j) + \sum_{i=1}^N h_i(\sigma_i)
\end{equation*}

\subsection{Pattern recognition}

\section{Protein folding}


\subsection{Homology modelling}
\subsection{\emph{Ab-initio} folding}

\section{Other predictors}



\part{My work} \label{part:work}

\chapter{Papers included in this thesis}

\todo[inline]{Provisional abstracts}

\section{Paper I}
Protein quality assessment is a long-standing problem in bioinformatics. For more than a decade we have developed state-of-art predictors by carefully selecting and optimising inputs to a machine learning method. The correlation has increased from 0.60 in ProQ to 0.81 in ProQ2 and 0.85 in ProQ3 mainly by adding a large set of carefully tuned descriptions of a protein. Here, we show that a substantial improvement can be obtained using exactly the same inputs as in ProQ2 or ProQ3 but replacing the support vector machine by a deep neural network. This improves the Pearson correlation to 0.90 (0.85 using ProQ2 input features).

\section{Paper II}

Accurate contact predictions can be used for predicting the structure of proteins. Until recently these methods were limited to very big protein families, decreasing their utility. However, recent progress by combining direct coupling analysis with machine learning methods has made it possible to predict accurate contact maps for smaller families. To what extent these predictions can be used to produce accurate models of the families is not known.

We present the PconsFold2 pipeline that uses contact predictions from PconsC3, the CONFOLD folding algorithm and model quality estimations to predict the structure of a protein. We show that the model quality estimation significantly increases the number of models that reliably can be identified. Finally, we apply PconsFold2 to 6379 Pfam families of unknown structure and find that PconsFold2 can, with an estimated 90\% specificity, predict the structure of up to 558 Pfam families of unknown structure. Out of these, 415 have not been reported before.

\section{Paper III}
Residue contact prediction was revolutionized recently by the introduction of direct coupling analysis (DCA). Further improvements, in particular for small families, have been obtained by the combination of DCA and deep learning methods. However, existing deep learning contact prediction methods often rely on a number of external programs and are therefore computationally expensive.


Here, we introduce a novel contact predictor, PconsC4, which performs on par with state of the art methods. PconsC4 is heavily optimized, does not use any external programs and therefore is significantly faster and easier to use than other methods.

\section{Paper IV}
 Proteins fold into complex structures that are crucial for their biological func-
tions. Experimental determination of protein structures is costly and therefore limited to a small
fraction of all known proteins. Hence, different computational structure prediction methods are
necessary for the modelling of the vast majority of all proteins. In most structure prediction
pipelines, the last step is to select the best available model and to estimate its accuracy. This
model quality estimation problem has been growing in importance during the last decade, and
progress is believed to be important for large scale modelling of proteins.
Current machine learning models trained to estimate the protein model quality suffer from
biases in the training set: multiple models of only a few targets, generated by a few methods.

 We propose a new methodology to train deep networks that leverages the structure of
the problem and takes advantage of some of this redundancies. We demonstrate its viability by
reaching results comparable with another state-of-the-art method, ProQ3D, trained and evaluated
on the same datasets, but employing only a small subset of the input features.
The proposed training strategy is applicable to other input features and datasets, and thus can
be applied to other programs.


\section{Paper V}

Methods to reliably estimate the accuracy of 3D models of proteins are both a fundamental part
of most protein folding pipelines and important for reliable identification of the best models when
multiple pipelines are used. Here, we describe the progress made from CASP12 to CASP13 in
the field of estimation of model accuracy (EMA) as seen from the progress of the most
successful methods in CASP13. We show small but clear progress, i.e. several methods
perform better than the best methods from CASP12 when tested on CASP13 EMA targets.


\backmatter
\chapter*{Acknowledgements}

\lettrine[lines=3, lhang=0.25, nindent=0em]{\color{Maroon}M}{any people have helped me during the years,}
and this space is to thank them.
I want to thank my supervisor, Arne Elofsson, for accepting me, his guidance 
\newgeometry{outer=24mm, inner=12mm}
\KOMAoptions{open=any}

{\small \raggedright
%	\printbibliography}
	\bibliography{references}}


\end{document}
