\chapter{Papers included in this thesis}

\todo[inline]{Provisional abstracts}

\section{Paper I}
Protein quality assessment is a long-standing problem in bioinformatics. For more than a decade we have developed state-of-art predictors by carefully selecting and optimising inputs to a machine learning method. The correlation has increased from 0.60 in ProQ to 0.81 in ProQ2 and 0.85 in ProQ3 mainly by adding a large set of carefully tuned descriptions of a protein. Here, we show that a substantial improvement can be obtained using exactly the same inputs as in ProQ2 or ProQ3 but replacing the support vector machine by a deep neural network. This improves the Pearson correlation to 0.90 (0.85 using ProQ2 input features).

\section{Paper II}

Accurate contact predictions can be used for predicting the structure of proteins. Until recently these methods were limited to very big protein families, decreasing their utility. However, recent progress by combining direct coupling analysis with machine learning methods has made it possible to predict accurate contact maps for smaller families. To what extent these predictions can be used to produce accurate models of the families is not known.

We present the PconsFold2 pipeline that uses contact predictions from PconsC3, the CONFOLD folding algorithm and model quality estimations to predict the structure of a protein. We show that the model quality estimation significantly increases the number of models that reliably can be identified. Finally, we apply PconsFold2 to 6379 Pfam families of unknown structure and find that PconsFold2 can, with an estimated 90\% specificity, predict the structure of up to 558 Pfam families of unknown structure. Out of these, 415 have not been reported before.

\section{Paper III}
Residue contact prediction was revolutionized recently by the introduction of direct coupling analysis (DCA). Further improvements, in particular for small families, have been obtained by the combination of DCA and deep learning methods. However, existing deep learning contact prediction methods often rely on a number of external programs and are therefore computationally expensive.


Here, we introduce a novel contact predictor, PconsC4, which performs on par with state of the art methods. PconsC4 is heavily optimized, does not use any external programs and therefore is significantly faster and easier to use than other methods.

\section{Paper IV}
 Proteins fold into complex structures that are crucial for their biological func-
tions. Experimental determination of protein structures is costly and therefore limited to a small
fraction of all known proteins. Hence, different computational structure prediction methods are
necessary for the modelling of the vast majority of all proteins. In most structure prediction
pipelines, the last step is to select the best available model and to estimate its accuracy. This
model quality estimation problem has been growing in importance during the last decade, and
progress is believed to be important for large scale modelling of proteins.
Current machine learning models trained to estimate the protein model quality suffer from
biases in the training set: multiple models of only a few targets, generated by a few methods.

 We propose a new methodology to train deep networks that leverages the structure of
the problem and takes advantage of some of this redundancies. We demonstrate its viability by
reaching results comparable with another state-of-the-art method, ProQ3D, trained and evaluated
on the same datasets, but employing only a small subset of the input features.
The proposed training strategy is applicable to other input features and datasets, and thus can
be applied to other programs.


\section{Paper V}

Methods to reliably estimate the accuracy of 3D models of proteins are both a fundamental part
of most protein folding pipelines and important for reliable identification of the best models when
multiple pipelines are used. Here, we describe the progress made from CASP12 to CASP13 in
the field of estimation of model accuracy (EMA) as seen from the progress of the most
successful methods in CASP13. We show small but clear progress, i.e. several methods
perform better than the best methods from CASP12 when tested on CASP13 EMA targets.