\chapter{Abstract}
\lettrine[lines=3, lhang=0.25, nindent=0em, findent=2pt]{\color{Maroon}P}{roteins are\ }
the basic molecular machines of the cell.
Experimental studies are usually labour-intensive and expensive; and in no few cases, fail without giving a result.
One of the aims of the field of Bioinformatics is to, when possible, replace these experiments with computational models.

The human impulse is to categorise and model everything based on hard and simple rules.
But Nature is messy and resists to adhere to these rules.
One of the solutions put forward was the use of machine learning: soft and complicated rules, reached by the computer after looking at the data.
The rules are ``soft" because they are probabilistic in nature, and ``complicated" because they are the result of analysing large datasets.

In the last decade, deep learning has revolutionised the field of machine learning, in particular in computer vision and speech recognition.
The primary reason for their success is the ability to train very flexible models, capable of capturing the complexity of the real world, on large amounts of data.
But the key that makes it feasible is the ability to leverage the structure in the data. 

This work presents the application of machine learning in general, and deep learning in particular, to several tasks in the field of protein structure prediction: contact prediction, \emph{ab-initio} modelling, model quality assessment.


