\chapter{Papers included in this thesis}

\section*{Paper I}
\begin{center}
	\textsc{ProQ3D: improved model quality assessments using deep learning}
\end{center}
\noindent
In this paper, we took the same features and dataset used to develop ProQ3, but replaced the machine learning algorithm with a multi-layer perceptron.
This allows us to make full use of the whole dataset, and boosted its performance, improving the correlation between predicted and true from 0.85 to 0.90.

\section*{Paper II}
\begin{center}
	\textsc{Large-scale structure prediction by improved contact predictions and model quality assessment.}
\end{center}

\noindent
Here we present a pipeline for contact-based \emph{ab-initio} protein structure prediction.
It predicts contacts with PconsC3, folds models with \CONFOLD, and selects and evaluates the accuracy using Pcons and ProQ3.
\newpage
\section*{Paper III}
\begin{center}
	\textsc{PconsC4: fast, accurate and hassle-free contact predictions}
\end{center}

\noindent
PconsC4 is a contact predictor designed to be fast and easy to use.
To accomplish this, the code is very optimised, requires only one \MSA, and does not depend on external predictors.
Therefore, it is easy to install, deploy, and apply in large-scale studies.


\section*{Paper IV}
\begin{center}
	\textsc{A novel training procedure to train deep networks in the assessment of the quality of protein models}
\end{center}

\noindent
The power of deep learning lies in its ability to leverage the inherent structure of the data.
In this work, we show how we can take this one step further, and bake the structure of the \emph{problem} in the architecture.
We demonstrate it is a viable alternative reaching or surpassing ProQ3D's performance on the same training and test sets, but using only a subset of the inputs.

\section*{Paper V}
\begin{center}
	\textsc{Estimation of model quality accuracy in CASP13}
\end{center}

\noindent
This paper presents the results on model quality assessment of the latest edition of the blind test Critical Assessment of (Protein) Structure Prediction, \CASP13,
and serves as independent validation for the results in Papers \textcolor{Maroon}{I} and \textcolor{Maroon}{IV}.
